\documentclass[onecolumn,preprint,superscriptaddress,nofootinbib,notitlepage,10pt,linenumbers]{revtex4-1}

\usepackage[perpage]{footmisc}
\usepackage{epstopdf}
\usepackage{epsfig}
\usepackage{color}
\usepackage{hyperref}
\usepackage{mathbbol}
%\usepackage{bookmark}
\usepackage{mathtools}
\usepackage{bbold}
\usepackage{amsmath,array}
\usepackage{amssymb}
\usepackage{slashed}
\usepackage{nicefrac}
\usepackage{tikz-feynman}
\usepackage{cjhebrew}
\usepackage{enumitem}

\setlength{\topmargin}{-1.0cm}
\setlength{\headheight}{0.1cm} \setlength{\footskip}{1.cm}
\setlength{\headsep}{.1cm}
\setlength{\textheight}{.8\paperheight}
\setlength{\textwidth}{.8\paperwidth}
\setlength{\oddsidemargin}{-.5cm}
\setlength{\evensidemargin}{0.0cm}
\setlength{\marginparwidth}{0.0cm}
\setlength{\marginparsep}{0.0cm}

\definecolor{blue}{HTML}{4169E1}
\definecolor{red}{HTML}{DC143C}
\definecolor{green}{HTML}{2E8B57}
\definecolor{black}{HTML}{000000}
\definecolor{g1}{HTML}{A9A9A9}
\definecolor{g2}{HTML}{696969}
\definecolor{g3}{HTML}{7F7F7F}
\definecolor{g4}{HTML}{D3D3D3}

\newcommand{\xdownarrow}[1]{%
  {\left\downarrow\vbox to #1{}\right.\kern-\nulldelimiterspace}
}

\newcommand{\black}[1]{\color{black}{#1}}
\newcommand{\red}[1]{\color{red}{#1}}
\newcommand{\blue}[1]{\color{blue}{#1}}
\newcommand{\green}[1]{\color{green}{#1}}
\newcommand{\caf}{\text{\cjRL{b}}}
\newcommand{\he}{${}^4$He}
\newcommand{\hes}{${}^3$He}
\newcommand{\tr}{${}^3$H}
\newcommand{\ls}{\ve{L}\cdot\ve{S}}
\newcommand{\eps}{\epsilon}
\newcommand{\as}{a_s}
\newcommand{\at}{a_t}
\newcommand{\ecm}{E_\textrm{\small c.m.}}
\newcommand{\dq}{\mbox{d\hspace{-.55em}$^-$}}
\newcommand{\mpis}{$m_\pi=137~${\small MeV}}
\newcommand{\mpim}{$m_\pi=450~${\small MeV}}
\newcommand{\mpil}{$m_\pi=806~${\small MeV}}
\newcommand{\muh}{\mu_{^3\text{\scriptsize He}}}
\newcommand{\mut}{\mu_{^3\text{\scriptsize H}}}
\newcommand{\mud}{\mu_\text{\scriptsize D}}
\newcommand{\pode}{\beta_{\text{\scriptsize D},\pm1}}
\newcommand{\poh}{\beta_{^3\text{\scriptsize He}}}
\newcommand{\pot}{\beta_{^3\text{\scriptsize H}}}
\newcommand{\com}[1]{{\scriptsize \sffamily \bfseries \color{red}{#1}}}
\newcommand{\eg}{\textit{e.g.}\;}
\newcommand{\ie}{\textit{i.e.}\;}
\newcommand{\cf}{\textit{c.f.}\;}
\newcommand{\be}{\begin{equation}}
\newcommand{\ee}{\end{equation}}
\newcommand{\la}{\label}
\newcommand{\ber}{\begin{eqnarray}}
\newcommand{\eer}{\end{eqnarray}}
\newcommand{\nn}{\nonumber}
\newcommand{\half}{\frac{1}{2}}
\newcommand{\thalf}{\nicefrac[]{3}{2}}
\newcommand{\bs}[1]{\ensuremath{\boldsymbol{#1}}}
\newcommand{\bea}{\begin{eqnarray}}
\newcommand{\eea}{\end{eqnarray}}
\newcommand{\beq}{\begin{align}}
\newcommand{\eeq}{\end{align}}
\newcommand{\bk}{\bs k}
\newcommand{\bt}{B_{^{3}\text{H}}}
\newcommand{\bh}{B_{^{3}\text{He}}}
\newcommand{\bd}{B_\text{D}}
\newcommand{\ba}{B_\alpha}
\newcommand{\rgm}{$\mathbb{R}$GM}
\newcommand{\ev}[1] {|\bra #1  \ket |^2}
\newcommand{\parg}[1] {\paragraph*{-\,\textit{#1}\,-}}
\newcommand{\nopi}{\pi\hspace{-6pt}/}
\newcommand{\ve}[1]{\ensuremath{\boldsymbol{#1}}}
\newcommand{\xvec}{\bs{x}}
\newcommand{\rvec}{\bs{r}}
\newcommand{\sgve}{\ensuremath{\boldsymbol{\sigma}}}
\newcommand{\tave}{\ensuremath{\boldsymbol{\tau}}}
\newcommand{\na}{\nabla}
\newcommand{\bra}[1] {\left\langle~#1~\right|}
\newcommand{\ket}[1] {\left|~#1~\right\rangle}
\newcommand{\bet}[1] {\left|#1\right|}
\newcommand{\overlap}[2] {\left\langle~#1~\left|~#2~\right.\right\rangle}
\newcommand{\me}[3] {\left\langle~#1~\left|~#2~\left|~#3~\right.\right.\right\rangle}
\newcommand{\redme}[3] {\left\langle~#1~\left|\left|~#2~\left|\left|~#3~\right.\right.\right.\right.\right\rangle}
\newcommand{\lam}[1]{$\Lambda=#1~$fm$^{-1}$}\newcommand{\tx}{\tilde{x}}
\newcommand{\eftnopi}{\mbox{EFT($\slashed{\pi}$)}}
\newcommand{\threej}[6]{\ensuremath{\begin{pmatrix}#1 & #2 & #3\\#4&#5&#6 \end{pmatrix}}}
\newcommand{\clg}[6]{\ensuremath{\left(\begin{array}{cccc|cc}#1 & #2 & #3 & #4 & #5 & #6 \end{array}\right)}}
\newcommand{\re}[1] {\mathcal{R}\left[#1\right]}
\newcommand{\im}[1] {\mathcal{I}\left[#1\right]}
\newcommand{\E}{\mathcal{E}}

\DefineFNsymbols*{lamportnostar}[math]{\dagger\ddagger\S\P\|{\dagger\dagger}{\ddagger\ddagger}}
\setfnsymbol{lamportnostar}
\renewcommand\thefootnote{\fnsymbol{footnote}}
\let\endtitlepage\relax

\begin{document}

\title{Lorentz and Siegert offer Compton their assistance.}
\author{Jean Luc Picard}
\email{jeanluc@1701.ncc}
\affiliation{Starfleet Academy, Fort Baker, San Francisco, Earth}
\date{\today}

\begin{abstract}
We study (elastic) Compton scattering off nuclei as a function of the
nucleon number. We consider processes with an initial state comprised of
photons, pions, protons, and neutrons at energies which are insufficient to
create particles that are not composites of these basic degrees of freedom.
The appropriate theory for amplitudes at momentum scales of the order of the pion mass
is constrained by Galilean, chiral, and $U(1)$-gauge symmetries.
The systematic ordering of the interactions thence admissible, allows us to
resolve the various correlations between observables in systems of different
size, kinematics, level of excitation, {\it etc.}. Thereby, we investigate to what
extent properties of large systems can be understood as consequences of the
two- and three-body interaction, but we also address the practical problem, \eg,
to assess the induced uncertainty in a few-body observable due to that in a
one-body quantity -- {\it To what extent does our poor knowledge about the neutron
polarizability affect predictions of deuteron and helium properties?}

An intriguing problem in its own right is the correct application of the
coupling of the two photons to the hadrons and the meson as specified and well
understood in {\it chiral perturbation theory} -- a relativistic quantum field theory --
in a solution of the non-relativistic few-nucleon problem.
\end{abstract}

%\maketitle


\paragraph{The ``LIT equation''}
\be\la{eq.liteq}
\left(\hat{H}_\text{nuclear}\underbrace{-E_0-\re{\sigma}-i~\im{\sigma}}_{:=-\E}\right)\Psi_\text{LIT}^{J^\pi m_j}=
\left[\hat{\mathcal{O}}_{Lm_L}\left\lbrace\bet{\ve{k}},\ve{j}_{v}\right\rbrace\otimes\Psi_0^{J_0^{\pi_0}}\right]^{J^\pi m_j}\;\;.
\ee
with
\begin{align}\la{eq.liteq.descr}
v(\text{ertex})\in&\left\lbrace~\ve{j}_o(\ve{x})=\ldots~,~\ve{j}_s(\ve{x})=\ldots~,~\ve{j}_{mec}(\ve{x})=\ldots~,~\ldots\right\rbrace\;\;;
\end{align}

\paragraph{The variational basis}
\be\la{eq.basis}
\Psi_\text{LIT}^{J^\pi m_j}=\sum\limits_nu_n~\phi^{J^\pi m_j}_n\;\;.
\ee
with
\begin{align}\la{eq.basis.descr}
\phi^{J^\pi m_j}_n=\left[\xi_{S_n}\otimes\mathcal{Y}_{l_n}(\ve{\rho})\right]^{J m_j}~e^{-\gamma_n\ve{\rho}^2}\;\;\;\;(\text{\ie, LS coupling})\;\;;
\end{align}

\paragraph{The matrix form of the ``LIT equation''}
\be\la{eq.liteq.mat}
\sum\limits_{s=1}^{N_\text{LIT}}\phi^{J^\pi m_j}_r\left(\hat{H}_\text{nuclear}-\E\right)\phi^{J^\pi m_j}_s~u_s=
\sum\limits_{n=1}^{N_0}\sum\limits_{m_L}~c_n~\underbrace{\clg{L}{m_L}{J_0}{m_j-m_L}{J}{m_j}}_{\blue{\texttt{enemb:ecce}}}~\phi^{J^\pi m_j}_r\hat{\mathcal{O}}_{Lm_L}\,\phi^{J_0^{\pi_0}(m_j-m_L)}_n\;\;.
\ee
with
\begin{align}\la{eq.liteq.mat.descr}
N_\text{LIT}~:&~\text{number of basis states used to expand the LIT state, \eg,} \Psi_\text{LIT}^{2^-}\;\;\;\;;\\
N_0\leq N_\text{LIT}~:&~\text{number of basis states used to expand the target, \eg, the deuteron;}\\
\end{align}

\paragraph{The matrix element}
\begin{align}\la{eq.me}
\phi^{J^\pi m_j}_m~\hat{\mathcal{O}}_{Lm_L}~\phi^{J_0^{\pi_0}m_{j_0}}_n:=&
\me{m;l_lS_lJ_lm_{j_l}}{\mathcal{A}~\mathcal{O}_{Lm_L}}{l_rS_rJ_rm_{j_r};n} \\
=&\underbrace{(-1)^{L-J_r+J_l}\frac{\clg{L}{m_L}{J_r}{m_{j_r}}{J_l}{m_{j_r}+m_L}}{\hat{J_l}}}_
{\blue{\texttt{enemb:600ff}}}
\underbrace{\redme{m;l_lS_lJ_l}{\mathcal{A}~\mathcal{O}_{L}}{l_rS_rJ_r;n}}_{\xdownarrow{.75cm}}\\
&\underbrace{\hat{J_r}\hat{J_l}\hat{L}
\left\lbrace\begin{array}{ccc}l^m_l & l^n_r & p\\ S^m_l & S^n_r & q \\ J_l & J_r & L\end{array}\right\rbrace}_{\blue{\texttt{enemb:ecce}}}
\sum_\text{dc}\sum_{\mathcal{P}\in\text{dc}}
\underbrace{\redme{m;l^m_l}{\mathcal{O}^o_p}{\mathcal{A}_\text{dc}l^n_r;n}}_{\red{\texttt{luise}}}\cdot
\underbrace{\redme{m;S^m_l}{\mathcal{O}^s_q}{\mathcal{A}_\mathcal{P}S^n_r;n}}_{\red{\texttt{obem}}}
\end{align}
with
\begin{align}\la{eq.liteq.mat.descr}
\hat{a}:=\sqrt{2a+1}\;\;;\\
\mathcal{A}=\sum_{\mathcal{P}\in\mathcal{S}_{A-1}}(-1)^{\text{sgn}(\mathcal{P})}\hat{\mathcal{P}}=\oplus_\text{dc}\\
\text{dc ~:~ double co-set}
\end{align}

\paragraph{The calculation}

\begin{enumerate}[label=(\roman*)]
\item Solve $$\hat{H}_\text{nuclear}~\Psi^{J_0^{\pi_0}}=E_0~\Psi^{J_0^{\pi_0}}$$ with ansatz
$$\Psi^{J_0^{\pi_0}}=\sum\limits_nc_n~\phi^{J_0^{\pi_0}}_n\;\;\;.$$
If $\hat{H}_\text{nuclear}$ is a spherical rank-0 operator -- a condition which most practical nuclear potentials satisfy --
$\Psi^{J_0^{\pi_0}}\neq f(m_{j_0})$. We obtain $\Psi^{J_0\red{J_0}}$, in practice.
\item Calculate
$$H_{rs}:=\me{\phi^{J^\pi}_r}{\hat{H}_\text{nuclear}}{\phi^{J^\pi}_s}\;\;\text{and}\;\;N_{rs}:=\overlap{\phi^{J^\pi}_r}{\phi^{J^\pi}_s}\;\;\;
\forall~|L-J_0|\leq J\leq|L+J_0|$$
\item Calculate
$$S^{Jm_j}_{rs}:=\me{\phi^{J^\pi m_j}_r}{\hat{\mathcal{O}}_{Lm_L}}{\phi^{J_0^{\pi_0}m_{j_0}}_s}\;\;,$$
and superimpose these matrix elements according to Eq.\eqref{eq.liteq.mat}
\be\la{eq.litinhomo}
S^{Jm_j}_r:=\sum\limits_{m_L}~c_n~\clg{L}{m_L}{J_0}{m_{j_0}-m_L}{J}{m_j}~S^{Jm_j}_{rn}\;\;\;.
\ee
\item Solve the (complex) linear matrix equation
\be\la{eq.liteq2}
\left(H_{rs}-\E N_{rs}\right)u^{Jm_j}_s=S_r
\ee
to obtain the LIT state
\be\la{eq.litstate}
\psi^{v(\text{ertex}),(\text{mu})L(\text{tipolarity})}_{J_{i(\text{nitial})/f(\text{inal})};J_{(\text{i})n(\text{termediate})}m_n}(k,\sigma)
=\blue{\psi^{v,L}_{J_0;Jm_j}(k,\sigma)}\black{:=
\Psi_\text{LIT}^{J^\pi m_j}\big(\underbrace{\bet{\ve{k}},v,L}_{\text{vertex}\atop\text{quantum numbers}};\underbrace{E_0,J_0}_{\text{initial/final-state}\atop\text{quantum numbers}};\re{\sigma},\im{\sigma}\big)\;\;\;.}
\ee
\item The inner product
\begin{align}\la{eq.litME}
\mathcal{L}_{v'L',vL}^{J_f,J_i;J}(k',k,\sigma)=&(-1)^{J-J_i+L-L'+v'}N_{J,\sigma}\sum_{m_j}
\underbrace{\overlap{\psi^{v',L'}_{J_f;Jm_j}(k',\sigma)}{\psi^{v,L}_{J_i;Jm_j}(k,\sigma)}}_{=\sum_{r,s}(u^{Jm_j}_r)^*u^{Jm_j}_s~N_{rs}}
\end{align}
\end{enumerate}

\newpage

\section{formulas and constants}

\begin{alignat}{2}
\text{(Wigner) 3-$j$ symbol:} &\hspace{1cm}& \threej{L}{S}{J}{m_l}{m_s}{-m_j}=&~(-1)^{L-S+m_j}~(2J+1)^{-\frac{1}{2}}~(Lm_l~Sm_s~\vert~LS~Jm_j)\\
\text{Matrix for single-axis rotation:} &\hspace{1cm}&   \mathcal{D}_{m^\prime,m}^{(j)}(0~\beta~0)\equiv&~d_{m^\prime,m}^{(j)}(\beta)\nonumber\\
&\hspace{1cm}& =&\left[\frac{(j+m^\prime)!(j-m^\prime)!}{(j+m)!(j-m)!}\right]^\frac{1}{2}\nonumber\\
&\hspace{1cm}&&~\cdot\sum_\sigma\left(j+m\atop j-m^\prime-\sigma\right)\left(j-m\atop \sigma\right)(-1)^{j-m^\prime-\sigma}\nonumber\\
&\hspace{1cm}&&~\cdot\left(\cos\frac{\beta}{2}\right)^{2\sigma+m+m^\prime}\left(\sin\frac{\beta}{2}\right)^{2j-2\sigma-m-m^\prime}
\end{alignat}

\bibliography{refs}

\end{document}